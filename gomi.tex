  シンタックスハイライトは,識別子であることは分かるが,その識別子が世の中でどの程度使われているものなのか,ということは分からない.
  そのため,libraryという識別子を誤ってlibralyなどと書いていても,プログラムとして動作すれば,綴りの誤りに気付かない.
  libralyのような綴り間違いの場合は,英語の辞書を用いると,変なことが分かるが,BufferedReaderと書くべきところを,BufferReaderと書くなど,英語として意味の通じるような間違いの場合は,辞書を使っても判断できない.
  また,組織内でのみ使われる単語は,一般的な辞書には載っていない.

  変な識別子を見つけるという他に,識別子がどのくらいの頻度で使われるものか分かると,プログラムを書いたり読んだりする際に役に立つと思われる.
  プログラムを書いたときに,よく使われるものと思って,書いた識別子の出現頻度が低いと,何か思い違いをしている可能性がある.
  たとえば,Rubyでライブラリをロードするには,requireというメソッドと,loadというメソッドがあるが,一般的に,requireはloadより多く使われる.
  requireでロードされたときにはライブラリは一度だけロードされるのに対し,loadは既にロード済みのライブラリでも再度読み込むため,requireのかわりにloadを使うとパフォーマンス上問題がある.
  単にライブラリをロードしようとしてloadと書くと,loadの出現確率は低いため,何か変だということに気付くことができる.

  識別子が一般的かどうかは,辞書では判定できない.
  世の中のソースコードを集めて,解析して,出現確率を調べることで,一般的かどうかが分かる.
